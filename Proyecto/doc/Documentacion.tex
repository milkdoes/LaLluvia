\documentclass{article}
\usepackage{graphicx} % Graficas.
\usepackage[utf8]{inputenc} % Uso de UTF.
\usepackage[hidelinks]{hyperref} % Indice con hipervinculos.
\usepackage{caption}
\usepackage{listings} % Copeo de codigo.
\usepackage{fullpage} % Uso de pagina completa.
\usepackage{color} % Color para codigo.

\renewcommand*\contentsname{Índice}

\begin{document}

\begin{figure}[!htb]
\minipage{0.32\textwidth}
	\includegraphics[width=\linewidth]{"img/LogoSEP".png}
\endminipage\hfill
\minipage{0.32\textwidth}
	\includegraphics[width=\linewidth]{"img/LogoTNM".PNG}
\endminipage\hfill
\minipage{0.32\textwidth}
	\includegraphics[width=\linewidth]{"img/LogoITT".png}
\endminipage\hfill
\end{figure}

\begingroup
\LARGE
\begin{verbatim}
Subdirección Académica
Departamento de Sistemas y Computación
Ingeniería en Sistemas Computacionales
Semestre: Enero - Junio 2017
Materia: Sistemas Programables (3SC8A)

Nombre del tema:
Documentacion proyecto

Nombre de los integrantes:
Salcedo Morales José Manuel (13211419)
Espinoza Covarrubias Silverio Alejandro (13211465)
Álvarez Corral Miguel Ángel (13211384)

Nombre del catedrático:
Ingeniero Luís Alberto Mitre Padilla

\end{verbatim}
\endgroup

\newpage
\tableofcontents

\newpage
\section{Introducción}

\section{Componentes utilizados}
\begin{itemize}
        \item Arduino
        \item Cables Jumper
        \item Fuente de alimentacion para arduino
\end{itemize}

\newpage
\section{Marco Teórico}
\begin{itemize}
	\item Arduino: Arduino se refiere a una plataforma o placa de electrónica de código abierto y al software utilizado para programarlo. Arduino está diseñado para hacer la electrónica más accesible a los artistas, diseñadores, aficionados y a cualquiera interesado en la creación de objetos interactivos o entornos. Un tablero de Arduino se puede comprar pre-ensamblado o, porque el diseño de hardware es de código abierto, construido a mano. De cualquier manera, los usuarios pueden adaptar las tablas a sus necesidades, así como actualizar y distribuir sus propias versiones.
\end{itemize}

Este proyecto ha sido realizado con los siguientes componenetes:

\begin{itemize}
	\item Arduino UNO
	\item Protoboard
	\item Sensor de húmedad en suelo
	\item Sensor de húmedad  y temperatura (DTH11)
	\item Sensor barométrico
	\item Fotorresistencia
	\item Relay 5VDC – 120 VAC
	\item Resistencia
	\item Diodo 1N4007
\end{itemize}

\subsection{Arduino UNO}

El Arduino Uno es una placa electrónica basada en el  ATmega328  (ficha técnica). Cuenta con 14 pines digitales de entrada / salida (de los cuales 6 se pueden utilizar como salidas PWM), 6 entradas analógicas, un 16  MHz  resonador de cerámica, de una conexión USB, un conector de alimentación, una cabecera ICSP, y un botón de reinicio.

\subsection{Protoboard}

Es una especia de tablero con orificios, en la cual se pueden insertar componentes electrónicos y cables para armar circuitos. Como su nombre lo indica, esta tableta sirve para experimentar con circuitos electrónicos, con lo que se asegura un buen funcionamiento del mismo. 

\subsection{Sensor de húmedad en suelo}

Este sensor de humedad puede leer la cantidad de humedad presente en el suelo que lo rodea. Es un sensor de baja tecnología, pero es ideal para el seguimiento de un jardín urbano. Se trata de una herramienta indispensable para un jardín de contacto.

Este sensor utiliza las dos sondas para pasar corriente a través del suelo, y luego  lee la resistencia para obtener el nivel de humedad. Más agua hace que el suelo conduzca la electricidad con mayor facilidad (menos resistencia), mientras que el suelo seco es un mal conductor de la electricidad (más resistencia).

Características:

\begin{itemize}
	\item Sensibilidad ajustable ajustando el potenciómetro digital (azul).
	\item Voltaje de operación: 3.3V ~ 5V
	\item Modo de salida dual, salida digital y salida analógica más precisa.
	\item Agujeros de montaje para una fácil instalación.
	\item Dimensiones PCB: 30mm * 16mm.
	\item Dimensiones de sonda: 60mm * 30mm.
	\item Indicador de energía. Indicador alimentación (rojo) e indicador de salida de conmutación digital (verde).
	\item El módulo tiene un amplificador LM393.
\end{itemize}

Defnición de pines:

\begin{itemize}
	\item VCC (5V)
	\item GND
	\item Interfaz de salida digital (0 y 1) 
	\item Interfaz de salida analógica AO
\end{itemize}

\subsection{Sensor de húmedad  y temperatura (DTH11)}
Tarjeta con Sensor DHT11 Humedad resistivo ideal para sistemas de medición climatológicos o para controles de temperatura y humedad. Este sensor además incluye un sensor interno de temperatura NTC. Este módulo tiene una gran relación señal a ruido ante la interferencia y es muy durable.
Especificaciones:

\begin{itemize}
	\item Sensor resistivo de húmedad
	\item Sensor de temperatura NTC
	\item Voltaje de alimentación: 5V
	\item SRango de temperatura: 0-50ºC
	\item Rango de humedad: 20-90{\%}RH
	\item Tamaño: 25 x 16 x 7 mm
	\item Peso: 9g
\end{itemize}


\subsection{Sensor barométrico}


\subsection{Fotorresistencia}

Una fotorresistencia es un componente electrónico cuya resistencia disminuye con el aumento de intensidad de luz incidente. Puede también ser llamado fotorresistor, fotoconductor, célula fotoeléctrica o resistor dependiente de la luz, cuyas siglas, LDR, se originan de su nombre en inglés light-dependent resistor. Su cuerpo está formado por una célula foto receptora y dos patillas. Su símbolo electrónico es:

\subsection{Relay 5VDC – 120 VAC}

El relé o relevador es un dispositivo electromecánico. Funciona como un interruptor controlado por un circuito eléctrico en el que por medio de una bobina y un electroimán, se acciona un juego de uno o varios contactos que permiten abrir o cerrar otros circuitos eléctricos independientes. Lo que hace la bobina es crear un campo magnético que lleva los contactos a establecer una conexión. El electroimán, por su parte, permite el cierre de los contactos. De esta forma, el relevador actúa como un interruptor que puede fomentar el paso de la corriente eléctrica o su interrupción. 

\subsection{Resistencia}

Se le denomina resistencia eléctrica a la oposición al flujo de electrones al moverse a través de un conductor. La unidad de resistencia en el Sistema Internacional es el ohmio, que se representa con la letra griega omega (Ω), en honor al físico alemán Georg Simon Ohm, quien descubrió el principio que ahora lleva su nombre.

\subsection{Diodo 1N4007}

Es uno de los diodos de una serie muy utilizados en infinidad de equipos electrónicos. Se utiliza principalmente para convertir la corriente alterna en directa. Su encapsulado es de tipo DO-41. 

\newpage
\section{Desarrollo}

\subsection{Instalacion}
\textbf{El simbolo ``/'' significa la raiz del proyecto.}

\subsubsection{Prerequisitos}

\begin{itemize}
	\item Entrar y correr el archivo de \textbf{/Proyecto/config/InstalacionPrerequisitos.sh} con \textbf{sudo bash InstalacionPrerequisitos.sh}
	\item Entrar a mysql con el comando \textbf{mysql -u root -p} (o un usuario distinto a root si se tiene) e ingresando la contrase\~na del usuario. Correr el archivo de configuracion de base de datos con \textbf{source /rutaRaizProyecto/Proyecto/db/CreacionBd.sql}.
	\item Copiar el folder \textbf{/Proyecto/web} a \textbf{/var/www/html/} y entrar a la pagina con (ejemplo): \textbf{http://localhost/web/}
\end{itemize}

\subsubsection{Corrida}
Para mantener una constante lectura de Arduino y subida de valores obtenidos a la base de datos, se tiene que ir al folder \textbf{/Proyecto/arduino/SistemaRiego} y correr el script de Python con \textbf{python LecturaArduino.py}.
\newline Para verificar la subida de datos se puede ir a la pagina web o correr los comandos dentro de mysql:
\begin{lstlisting}[language=bash]
mysql> use sistema_riego;
mysql> select * from dato;
\end{lstlisting}

\subsection{Imagenes}

\newpage
\section{Conclusión}

\renewcommand\refname{Referencias}
\begin{thebibliography}{1}
	\bibitem{Arduino} What is Arduino? - Definition from Techopedia. (n.d.). Retrieved March 26, 2017, from https://www.techopedia.com/definition/27874/arduino
	\bibitem{Sensor de húmedad en suelo} Electronilab from https://electronilab.co/tienda/sensor-de-humedad-de-suelo-higrometro/
	\bibitem{Sensor de húmedad  y temperatura (DTH11)} HETPRO from https://hetpro-store.com/sensor-dht11-humedad/?utm_source=google&utm_medium=cpc&utm_campaign=Product&utm_term=SDTH0011&utm_content=SDTH0011&gclid=CjwKEAjwu4_JBRDpgs2RwsCbt1MSJABOY8an_YIe39bwPw9nGCoGn0nhwnEDJa0ecuyz6HhjDzsVNRoCX-7w_wcB
\end{thebibliography}

\end{document}
